\chapter{Posicionamiento de pacientes virtuales} 
\label{cap:posing}


En este capítulo, se propone un método que permite adaptar \new{la anatomía, interna y/o externa, de un modelo anatómico virtual a cualquier pose requerida}\del{modelos virtuales anatómicos, con estructuras tanto externas como internas, desde una posición inicial hasta una pose final.} \todo{He reestructurado el apartado} \new{Tal y como se explica en la capítulo de introducción \ref{cap:intro}, esta tesis se ha desarrollado en el contexto del proyecto \ac{RASimAs}, financiado por el 7º programa marco de la Unión Europea. Uno de los objetivos principales del proyecto fue crear un simulador de anestesia regional que permitiese entrenar a los futuros médicos. 
Este simulador debía ofrecer una amplia base de pacientes virtuales, de forma que el anestesista se pudiese enfrentar a una gran diversidad de variedades anatómicas durante su aprendizaje. Dependiendo del nervio que se desee bloquear, el procedimiento requiere una pose especifica del paciente. Los algoritmos, que aquí se describen, se diseñaron para adaptar los modelos anatómicos de la base de datos de pacientes a las poses requeridas en las distintas versiones del procedimiento. Dada su importancia, \ac{RASimAs} dedicó una tarea completa a solucionar esté problema (ver sección \ref{art:rasimas}). 
De esta forma, los requisitos que guiaron su diseño vienen impuestos por las necesidades del proyecto (ver sección \ref{posing:req}). Todos los algoritmos propuestos se integraron en la herramienta \ac{TPTVPH}, desarrollada completamente en el contexto de esta tesis y que formaba parte del \emph{suite de aplicaciones}}\footnote{Paquete de aplicaciones desarrolladas con distintos objetivos y capaces de cooperar entre sí.} \new{del módulo de creación de pacientes virtuales de  \ac{RASimAs}.}

\del{Como se puede leer en el capítulo de introducción \ref{cap:intro}, el objetivo de esta tesis viene influenciado por el proyecto europeo \ac{RASimAs} cuya finalidad de crear un simulador de anestesia que pueda entrenar con una base de datos de pacientes virtuales consiguiendo variabilidad anatómica adecuada para practicar el procedimiento.
El proyecto europeo dedica una tarea completa a solucionar la problemática de poder reposicionar modelos anatómicos con el objetivo de crear una base de datos con una gran variabilidad anatómica (ver sección \ref{art:rasimas}).}

\todo{No está mal escrito, pero es redundante con los objetivos de RASimAs}
\del{En el contexto de entrenamiento médico, es fundamental que los usuarios se enfrenten a la mayor cantidad de situaciones posibles. 
Es por ello, que se ha propuesto un método que permita la creación de un conjunto de datos extenso de forma automática o semi-automática y a la vez pueda ser utilizada en cualquier simulador de realidad virtual.}

\new{El conjunto de algoritmos, que se han diseñado a lo largo de este trabajo de tesis, extienden la animación esqueletal clásica a la animación de los tejidos internos de un modelo virtual. Al tratarse de un método puramente geométrico ofrece, con un coste computacional bajo, resultados plausibles y estables. Otro de los motivos principales por los que se escogió un método geométrico frente a un método basado en modelos fiscos, es la posibilidad de manipular modelos anatómicos incompletos o de los que no se dispone de propiedades mecánicas. Estas características, hacen que su uso sea adecuado adecuado en otras aplicaciones, incluso más allá del ámbito médico.  En el mundo del ocio interactivo, especialmente en el sector de los videojuegos, coste computacional, la estabilidad, la plausibilidad de los resultados y la posibilidad de aplicarlo a estructuras anatómicas incompletas, son características más importantes que la precisión física de las transformaciones. }

\todo{hay que dar mas peso a lo de los modelos incompletos que a la interactividad. En el caso de rasimas es más importante. }
\del{
Con este propósito, se han diseñado y adaptado diferentes algoritmos de animación esqueletal con el objetivo de animar tejidos internos y externos de un paciente virtual. Se ha elegido utilizar métodos geométricos que permiten la creación de resultados plausibles y a la vez tienen la ventaja de que permiten una deformación interactiva. Esta ventaja frente a los métodos basados en físicos dan lugar a que el algoritmo presentado en este capítulo pueda incluirse en cualquier simulador que requiera de deformación en tiempo real. Otra ventaja que tiene el método propuesto es la posibilidad de manipular modelos anatómicos incompletos o que no se disponen de sus propiedades mecánicas.
}

\section{Requisitos de diseño}
\label{posing:req}
\todo{De manera ideal no son los requisitos nos los fijan los socios. Sino los objetivos del proyecto. }
\del{Al formar parte de un proyecto donde intervienen más grupos de investigación, la tarea específica viene acompañada de una serie de requisitos que necesitan ser respetados, pero al mismo tiempo es la principal motivación que se encuentra al formular una solución innovadora:}
\new{En las etapas iniciales del proyecto \ac{RASimAs} se establecieron los objetivos del modulo de creación de pacientes virtuales, del que la herramienta \ac{TPTVPH} formaba parte. A partir de dichos objetivos se definieron los requisitos de \ac{TPTVPH} y por ende de los algoritmos que debían ser diseñados.}


\todo {0.No está escrito como unos requisitos. Mezclas requisitos con decisiones de diseño!!!
1. El enfoque geométrico no es una recomendación es una consecuencia de los objetivos
2. Realmente en el estado del arte, se ha demostrado que el enfoque geométrico es menos sensible a la calidad de datos de entrada?????. 
3. a que te refieres con tipos de datos. Realmente te refieres a modelos incompletos. }
\begin{itemize}
    \item \new{El proceso de selección de poses deberá ser guiado y supervisado por un experto para garantizar la validez de los modelos generados. Dado el coste económico del tiempo de un experto, la herramienta deberá tener un interfaz sencillo y los algoritmos deberán ejecutarse en modelos complejos con tasas de refresco interactivas.}
    \item{\new{Los datos de entrada se obtendrán promediando datos de diferentes pacientes reales y registrándolos sobre el modelo de \emph{Zygote}\cite{XXX}. Los datos de pacientes reales provendrán distintas técnicas de imagen médicas (US, MRI, TAC\todo{ponlo en notación de acrónimo}). Dado que no existe ninguna técnica de imagen capaz de capturar todas las estructuras anatómicas internas, los pacientes virtuales generados no dispondrán de modelos de todos los tejidos internos relevantes. Solo se garantiza que se dispondrá de modelos adecuados de piel y huesos. }}
    \item \new{En el momento de realizar el posicionamiento del paciente virtual, no se conocerán las propiedades mecánicas de las distintas estructuras anatómicas que lo componen.}
    \item \new{Los algoritmos propuestos deberán evitar que se produzcan autocolisiones o colisiones entre los distintos tejidos, dado que esto podría provocar fallos en los algunos módulos del simulador. El modulo de simulación de \ac{US} es especialmente sensible a colisiones y autocolisiones.}
     \item \del{Se recomienda un enfoque geométrico ya que, en fases previas al proyecto, se consideró la posibilidad de utilizar un método basado en físicas para la tarea en cuestión. Como se ha podido comprobar en el estado del arte \ref{art:animation}, un enfoque geométrico es menos dependiente de la calidad de los datos de entrada, obtenidos por etapas anteriores en este proyecto. Un modelo basado en físicas necesitaría una descripción completa de todos los tejidos; sin embargo, el método geométrico puede ser útil si estos datos no están disponibles.}
    \item  \del{\ac{TPTVPH} (ver sección \ref{art:rasimas}) podrá ser capaz de trabajar con cualquier tipo de datos que procedan de un paciente virtual. Como no se puede asegurar que todos los tejidos se encuentren disponibles, el algoritmo deberá ser lo máximo flexible posible para manejarlos.}
    \item \todo{Te lo he rescrito porque creo que me suena raro. Creo que no resaltas lo que es realmente importante. }\del{Para mantener la calidad del modelo anatómico de las siguientes etapas y tareas dependientes de esta (p. ej. el módulo de \ac{US}), el resultado de este algoritmo no deberá generar colisiones adicionales indeseables entre los tejidos y tener igual o menos colisiones que las existentes al llegar como entrada.}
    
    \item \del{Un conjunto de poses podrá ser definido para acelerar el proceso completo. \ac{TPTVPH} podrá ser completamente automática.}\todo{Poco relevante, lo puedes llevar al discusión como caracteristica de la herramienta. }
    \item \todo{He usado este objetiv para meter la importancia de la interactividad.}\del{Aun así, el proceso de posicionamiento tendrá que ser supervisado. Profesionales cualificados definirán un conjunto genérico de posturas para el procedimiento de \ac{RA}. Un método semiautomático será desarrollado para facilitar la consecución de la validez y calidad de los modelos de pacientes virtuales.}
    \item \del{El usuario podrá modificar la postura del \ac{VPH} a través de una interfaz 3D interactiva \del{con la intención de reducir el procedimiento completo}. El algoritmo deberá ejecutar en tiempo real para mejorar la experiencia de uso. }
    

\end{itemize}


\todo{No puede haber solo una subsección!!!}
%\subsection{Requisitos del posicionador de pacientes virtuales}

\todo{Esto se va al final de capítulo en la discusión. Lo vamos a meter con las limitaciones de la técnica. }
\del{Aunque se ha intentado diseñar lo más flexible posible y minimizar el número de tejidos necesarios, se ha definido los siguientes requisitos mínimos:}
\begin{itemize}
    \item \del{Las estructuras mínimas que necesita el algoritmo son la piel y los huesos. Estos tejidos son los más habituales que se pueden capturar en muchas de las imágenes médicas, por tanto, servirán para guiar el proceso e identificar algunos puntos claves para el correcto funcionamiento del algoritmo.}
    
    \item \del{Debido a que la extracción de los tejidos desde imágenes médicas no es perfecta, se permiten que haya auto colisiones y colisiones entre tejidos. Aun así, se espera que los diferentes tejidos anatómicos no sobresalgan de la piel, ya que el algoritmo no está diseñado para resolver colisiones, además de no generar colisiones adicionales. Aunque aquellos tejidos que traspasen la piel son irreales, se tratarán por igual aunque no se puede asegurar una deformación realista.}
    
    \item \del{El algoritmo utilizará un esqueleto virtual para definir los movimientos de las articulaciones. Este esqueleto virtual no se puede confundir con el esqueleto real del paciente que son tejidos pertenecientes al \ac{VPH}. El esqueleto virtual es una abstracción matemática que es usada para definir puntos de rotación para las diferentes articulaciones del modelo. Estos puntos son llamados \ac{joints} y están organizados jerárquicamente (ver sección  \ref{art:virtualskel}). Se utilizará la información que proporciona los tejidos óseos para construir un esqueleto virtual adecuado al modelo de entrada.}
\end{itemize}

\todo{Si te sobra tiempo, puede describir las ventajas de los métodos fiscos sobre los geométricos. Coste computacional, estabilidad, flexibilidad. Por otro lado, los cambios de poses requieren grandes movimientos que para ser capturados de forma precisa necesitan que se tengan en cuenta numerosos y no faciles de simular comportamientos (deformaciones elásticas, interacciones:  deslizamientos, contactos). Requieren modelos completos y bien caracterizados... }
\new{Como consecuencia de estos requisitos, se decidió desechar opción de trabajar con métodos basados en modelos físicos y se opto por diseñar una técnica geométrica.} En las siguientes secciones se describirá el método propuesto.